\documentclass[12pt]{article}
\usepackage{fontspec}
\defaultfontfeatures{Ligatures=TeX}
\setmainfont{Times New Roman}
\usepackage{geometry}
\usepackage{setspace}
\usepackage{enumitem}
\usepackage[hidelinks]{hyperref}
\usepackage{titlesec}
\usepackage{nowidow}

% Widow and orphan control: minimum 3 lines per page
\setnowidow[4]
\setnoclub[4]
\widowpenalty=10000
\clubpenalty=10000
\brokenpenalty=10000
\interlinepenalty=1000 % Discourage breaks within paragraphs

% Set page geometry for 25-mm (1-inch) margins on A4 paper
\geometry{a4paper, margin=25mm}

% Double spacing
\doublespacing

% Section formatting
\titleformat{\section}{\large\bfseries}{\thesection}{1em}{}
\titleformat{\subsection}{\normalsize\bfseries}{\thesubsection}{1em}{}

% Title, author, and date
\title{The Dominant Themes and Intersecting Motifs \\ in the Book of Hebrews}
\author{Warwick Allen}
\date{28 June 2025}

\begin{document}


\maketitle

\begin{abstract}
The Book of Hebrews is a theologically rich New Testament epistle aimed at
encouraging early Jewish Christians to remain steadfast in their faith amidst
pressures to revert to Judaism. Six themes dominate the book: Christ’s
supremacy, the new covenant, Christ’s priesthood, faith and perseverance, the
call to holiness and worship, and the warning against rejecting Christ’s
sufficient atonement. These are unified by four interwoven motifs: rest,
covenant fulfilment, access to God’s presence, and perfection.

Hebrews extensively relies on the Old Testament, with 70--90\% of its content
engaging with the Old Testament. This grounds its claims, using Old Testament
scriptures to demonstrate Christ’s fulfilment of God’s redemptive plan.

The theme of Christ’s supremacy presents Him as superior to angels, Moses, and
the Levitical priesthood, fulfilling Old Testament promises and leading
believers into God’s rest. The new covenant, superior to the Mosaic Law, offers
forgiveness and direct access to God, completing Old Testament shadows and
enabling rest through faith. Christ’s eternal priesthood in the order of
Melchizedek surpasses the Levitical system, securing access to God’s presence
and perfecting believers. Faith and perseverance are urged through Old
Testament examples, with unbelief warned as barring entry to God’s rest. The
call to holiness and worship reflects gratitude for the new covenant, requiring
a life aligned with God’s presence. The warning against rejecting Christ’s
atonement emphasises the peril of relying on human works or the old sacrificial
system, which jeopardises redemption and rest.

The motifs unify these themes: rest symbolises salvation’s goal, covenant
fulfilment highlights Christ’s completion of Old Testament promises, access to
God’s presence emphasises relational intimacy, and perfection underscores
salvation’s completion. Each motif, rooted in Old Testament imagery, reinforces
Christ’s supremacy, the new covenant’s efficacy, and the necessity of a faith
kept pure from syncretism with any form of works-based salvation. Together, they
create a cohesive narrative, exhorting believers to trust fully in Christ’s
finished work to enter God’s rest, access His presence, and receive perfected
salvation, fulfilling the Old Testament’s promises.
\end{abstract}

\section{Introduction}
The Book of Hebrews, a profound epistle in the New Testament, is a theological
masterpiece crafted to encourage early Jewish Christians to remain steadfast in
their faith in Jesus Christ amidst pressures to revert to Judaism. After
discussing the book’s Old Testament foundation, this essay explores the six
most dominant themes in Hebrews---namely, the supremacy of Christ, the new
covenant, the priesthood of Christ, faith and perseverance, the call to
holiness and worship, and the critical warning against failing to trust fully
in Christ’s sufficient atonement. It also examines how the motifs of rest,
covenant fulfilment, access to God’s presence, and perfection interweave these
themes, serving as unifying threads that underscore the book’s message of
salvation through faith in Christ alone.

\section{Reliance on the Old Testament}
The Book of Hebrews heavily relies on the Old Testament to construct its
theological arguments, grounding its claims about Christ in the scriptures
familiar to its Jewish Christian audience. As John Calvin observes, ``The
Epistle to the Hebrews is a kind of commentary on the Old Testament, for it
explains how the figures and shadows of the Law are fulfiled in Christ, and
how the promises given to the fathers are accomplished in Him'' (Calvin 1853,
7). Approximately 30--40\% of Hebrews consists of direct Old Testament
quotations or explicit references, with an additional 40--50\% indirectly
referencing Old Testament concepts through themes, typology, or theological
frameworks, making roughly 70--90\% of the book engaged with the Old Testament
in some form. F.F. Bruce notes, ``The writer of Hebrews is saturated with the
Old Testament; he quotes it directly, alludes to it constantly, and assumes his
readers’ familiarity with its texts and themes to demonstrate the superiority
of Christ and His new covenant'' (Bruce 1964, 14). Hebrews includes at least 29
direct quotations from the Septuagint, drawn from texts like the Psalms,
Jeremiah, and Leviticus (e.g., Ps 95 in Heb 3:7--11; Jer 31:31--34 in Heb
8:8--12). These quotations, often introduced with phrases like ``as it is
written'' or ``the Holy Spirit says,'' emphasise their authoritative role.
Beyond direct quotes, Hebrews frequently alludes to Old Testament figures
(e.g., Melchizedek in Heb 7; Moses in Heb 3) and events (e.g., the wilderness
wandering in Heb 3--4), using them as types or shadows fulfilled in Christ.
Indirect references permeate the book’s discussions of the Law, sacrifices, and
priesthood, which assume familiarity with Old Testament institutions (e.g., Lev
16 in Heb 9). This pervasive Old Testament engagement anchors Hebrews’ argument
that Christ fulfills and surpasses the old covenant, making the Old Testament
indispensable to its message.

\section{The Supremacy of Christ}
At the heart of Hebrews lies the theme of Christ’s supremacy, portraying Him as
superior to all figures and institutions of the Old Testament. Hebrews 1:1--4
establishes Jesus as the eternal Son of God, the ultimate revelation who
surpasses angels, Moses, and the Levitical priesthood, fulfilling Old Testament
prophecies (e.g., Ps 2:7; 2 Sam 7:14). Charles Spurgeon declares, ``Jesus
Christ is set forth in Hebrews as above all angels, above Moses, above Aaron,
above all the shadows of the law, because He is the substance, the brightness
of the Father’s glory, and the express image of His person'' (Spurgeon 1871,
1012). As the creator and sustainer of all things, seated at God’s right hand
(Heb 1:3), Christ is the sole mediator of salvation. The motif of rest ties
directly to this theme, as Christ leads believers into God’s true rest,
surpassing the temporary rest of the Promised Land under Joshua (Heb 4:8--10,
referencing Ps 95). Covenant fulfilment underscores Christ’s role as the
culmination of Old Testament promises (Heb 1:1--2), as Augustine notes,
``Christ is the end of the law, the prophets, and the patriarchs; in Him all
things are fulfilled'' (Augustine 1887, 14.24). Access to God’s presence is
enabled by His exalted position (Heb 4:16), and perfection highlights Christ as
the pioneer of salvation, made ``perfect through suffering'' (Heb 2:10,
alluding to Old Testament suffering servant imagery).

Flowing naturally from Christ’s supremacy is the establishment of a new
covenant, for His exalted status not only fulfils the promises of the old but
inaugurates a superior covenantal relationship that renders the former obsolete.

\section{The New Covenant}
Hebrews contrasts the old covenant, rooted in the Mosaic Law, with the new
covenant established through Christ’s sacrifice. This new covenant is superior,
offering forgiveness and direct access to God (Heb 8:6--13, quoting Jer
31:31--34). John Owen affirms, ``The new covenant, as foretold by Jeremiah and
established by Christ, is the perfection of God’s promises, whereby sins are
forgiven, and the law is written on the heart, making the old covenant
obsolete'' (Owen 1854, 4:123). Hebrews 9:15 describes Christ as its mediator,
replacing the repetitive sacrifices of the old system (Lev 16) with His
once-for-all atonement. The motif of rest connects here, as the new covenant
enables believers to enter God’s rest through faith (Heb 4:3--5). Covenant
fulfilment is central, as the new covenant completes the Old Testament’s
temporary provisions (Heb 8:13). Access to God’s presence is a hallmark,
symbolised by the torn temple veil (Heb 10:19--22, referencing Old Testament
sanctuary imagery). Perfection is achieved through forgiveness, perfecting
believers’ consciences where the old covenant failed (Heb 10:1--4, 14), as R.C.
Sproul notes, ``The new covenant in Hebrews is the fulfilment of Jeremiah’s
prophecy, where Christ’s once-for-all sacrifice replaces the repeated offerings
of the old covenant'' (Sproul 1985, 45).

\subsection{Debates over Covenant Continuity and Discontinuity}
The portrayal of the new covenant in Hebrews has sparked significant scholarly
debate regarding its relationship to the old covenant. This debate, which
centres on the extent of continuity and discontinuity between the two covenants,
is crucial for understanding how Hebrews presents the new covenant as superior
and how this superiority relates to the epistle’s broader themes and motifs.
While the previous section established the new covenant’s role in offering
forgiveness and direct access to God, the question remains: does this new
covenant represent a radical break from the old, or does it fulfil and perfect
what the old covenant foreshadowed?

Hebrews itself provides seemingly contrasting signals. On one hand, Hebrews 8:13
declares that ``by calling this covenant \lq{}new,\rq\ he has made the first one
obsolete; and what is obsolete and outdated will soon disappear,'' suggesting a
sharp discontinuity. This perspective is echoed by scholars like Thomas
Schreiner, who argues that Hebrews presents a theology of covenantal
displacement, in which the Mosaic covenant becomes nullified in light of
Christ’s superior priesthood and sacrifice (Schreiner 2017, 204). Such a view
underscores the radical nature of Christ’s work, rendering the old covenant’s
sacrificial system entirely obsolete and emphasising the urgency of embracing
the new covenant fully. This interpretation aligns with Hebrews’ stark warnings
against reverting to the old system (e.g., Hebrews 10:26--31), as doing so would
mean rejecting the only sufficient means of atonement.

On the other hand, Hebrews also portrays the new covenant as the fulfilment of
promises embedded in the old covenant, suggesting a deep continuity. Peter
O’Brien notes that while the old covenant is depicted as insufficient for
salvation, it is never discarded as irrelevant. Instead, the new covenant is
presented as the divinely intended consummation of the old, particularly through
the prophecy of Jeremiah 31:31--34, which Hebrews quotes at length in 8:8--12
(O’Brien 2010, 296). This motif of fulfilment is further evident in Hebrews
10:1, where the law is described as ``only a shadow of the good things that are
coming---not the realities themselves.'' Thus, the new covenant does not abolish
the old arbitrarily but brings its sacrificial, priestly, and prophetic elements
to their intended completion in Christ. John Calvin captures this dual
perspective, stating, ``The old covenant was not contrary to the gospel, but was
its cradle; in Christ, the shadows are removed and the truth shines forth''
(Calvin 1853, 45).

This scholarly debate is not merely academic; it profoundly shapes how we
understand the epistle’s dominant themes and intersecting motifs. If the new
covenant is seen as a sharp discontinuity, it highlights the transformative
nature of Christ’s work and the obsolescence of the old system, reinforcing the
theme of Christ’s supremacy and the peril of rejecting His sufficient atonement.
This perspective also sharpens the motif of rest, as entry into God’s rest
becomes exclusively tied to faith in Christ rather than adherence to the old
covenant’s practices. Conversely, viewing the new covenant as a deep continuity
enriches our understanding of motifs like covenant fulfilment and perfection. It
shows how the old covenant’s temporary provisions—such as the Levitical
priesthood and repeated sacrifices—find their substance and completion in
Christ’s eternal priesthood and once-for-all sacrifice. This fulfilment motif
underscores that Christ does not discard the old covenant but perfects it,
achieving what it could only foreshadow.

Moreover, the way one interprets the relationship between the covenants
influences how believers understand their identity and practice under the new
covenant. A discontinuity view might lead to a greater emphasis on the
distinctiveness of Christian worship and ethics, free from the old covenant’s
rituals. In contrast, a continuity view could encourage believers to see their
faith as the true realisation of Old Testament worship, with Christ as the
ultimate High Priest and sacrifice. This perspective deepens the call to
holiness and worship (Hebrews 12:14, 28), as believers are invited to live in
grateful response to the fulfilled promises of God.

Ultimately, while Hebrews clearly presents the new covenant as superior and the
old as obsolete, the epistle also weaves a narrative of fulfilment that honours
the old covenant’s role in God’s redemptive plan. This nuanced interplay between
continuity and discontinuity enriches our understanding of the epistle’s themes,
particularly the supremacy of Christ and the perfection He achieves for
believers. It also reinforces the urgency of the warning against rejecting
Christ’s atonement, as to spurn the fulfilled covenant is to forfeit the very
rest, access, and perfection that the old covenant could never provide.

\section{The Priesthood of Christ}
Christ’s role as the eternal High Priest in the order of Melchizedek, surpassing
the Levitical priesthood, is a dominant theme. Hebrews 7:23--28, referencing Ps
110:4, highlights His eternal priesthood, with His single sacrifice perfecting
believers forever (Heb 10:11--14). Philip Edgcumbe Hughes states, ``The
priesthood of Melchizedek, as expounded in Hebrews, reveals Christ as the
eternal priest who, by His own blood, secures an eternal redemption'' (Hughes
1977, 258). This priesthood grants access to God’s presence (Heb 4:14--16),
central to the motif of rest, as the \textit{sabbatismos} of Heb 4:9--10
reflects God’s Sabbath rest, shared through Christ’s priestly work (Gen 2:2).

Athanasius’s theology of the incarnation offers a compelling lens through which
to view Christ’s priesthood in Hebrews. By becoming fully human while remaining
fully divine, Christ unites God and humanity, serving as both the eternal priest
and the perfect sacrifice. As Athanasius notes, ``He Himself is the sacrifice,
and He Himself is the priest,'' highlighting how Christ’s dual role achieves
what the Levitical system could not: a once-for-all atonement that purifies
humanity’s conscience and secures eternal intercession. This perspective
illuminates the motifs of perfection and access to God’s presence, as Christ’s
divine-human identity ensures His priesthood’s permanence and opens the way for
believers to approach God directly. Athanasius adds, ``Christ, as the eternal
High Priest after the order of Melchizedek, has offered Himself as the perfect
sacrifice, entering the heavenly sanctuary to intercede for us forever''
(Athanasius 2011, 9.2).

Covenant fulfilment is evident, as Christ’s priesthood completes the Aaronic
system (Heb 7:11--22). Access to God’s presence is secured by His mediation (Heb
10:19), and perfection is achieved, as His priesthood makes believers complete
(Heb 7:19, 28).

\section{Faith and Perseverance}
Hebrews emphasises faith and perseverance, particularly in trials. Chapter 11
recounts Old Testament figures who, despite significant failures and moral
shortcomings, exemplified enduring faith, encouraging believers to persevere
(Heb 10:19--39). Martin Luther observes, ``Faith, as Hebrews 11 teaches, is the
substance of things hoped for, and by it the saints of old trusted God’s
promises, looking forward to Christ'' (Luther 1968, 26). The motif of rest is
linked, as Heb 3:12--19, citing Ps 95, warns that unbelief prevents entry into
God’s rest, while Heb 4:2--3 assures that faith secures it. Abraham’s faith
(Hebrews 11:8-10) anticipates the rest fulfilled in Christ (Hebrews 4:9),
illustrating perseverance toward God’s promise. John Wesley adds,
``The faith of the patriarchs, as recounted in Hebrews, is our example to
persevere in trusting Christ, for without faith it is impossible to please
God'' (Wesley 1755, Heb 11:6). Covenant fulfilment connects, as faith in
Christ’s work secures Old Testament promises (Heb 11:39--40). Access to God’s
presence is accessed through faith (Heb 4:16), and perseverance ensures
continued access (Heb 10:22). Perfection is received by faith, as believers
await completion through Christ (Heb 11:40).

\section{The Call to Holiness and Worship}
Hebrews urges believers to embrace holiness and worship as vital responses to
Christ’s redemptive work. Hebrews 12:14 exhorts, ``be holy; without holiness no
one will see the Lord,'' while Hebrews 12:28-29 calls for worship ``with
reverence and awe.'' These commands are not mere ideals but practical
expressions of faith.  Jonathan Edwards captures this, noting, ``True holiness,
as urged in Hebrews, is the fruit of faith in Christ’s redemption, whereby we
live to God’s glory and worship Him in reverence'' (Edwards 1959, 2.3).

\emph{Holiness} entails ethical living, compassion, and peacemaking.
Practically, believers can reflect this by making honest choices at work,
showing kindness to the marginalised, or mediating disputes with grace (Hebrews
12:14). This flows from Christ’s sanctifying sacrifice (Hebrews 10:14). R.C.
Sproul underscores its urgency: ``Holiness is not optional for the Christian;
Hebrews 12:14 declares it essential to see the Lord'' (Sproul 1985, 112).
Without it, we cannot fully enter God’s presence.

\emph{Worship}, meanwhile, transcends rituals, permeating daily life through
gratitude, service, and awe. Simple acts—thanking God for a meal, serving a
neighbour, or marveling at a sunset—become offerings of praise (Hebrews
13:15-16). Engaging Scripture or creation deepens reverence (Hebrews 12:28).

These practices interweave with Hebrews’ motifs: holiness and worship bring rest
(Hebrews 4:9), fulfil the new covenant (Hebrews 8:10-12), grant access to God
(Hebrews 12:14), and reflect Christ’s perfection (Hebrews 10:14). Yet, they stem
from grace, not obligation (Hebrews 10:10). Edwards and Sproul affirm that
holiness and worship, rooted in Christ’s sufficiency, transform believers’
lives, fostering peace, integrity, and intimacy with God. But failing to
recognise that holiness and acts of worship stem from grace and are rooted in
Christ’s sufficiency, and thinking that they somehow help towards our
redemption, is, in fact, rejecting the sufficiency of Christ's atonement.

\section{Warning Against Rejecting Christ’s Sufficient Atonement}
The warning against failing to accept Christ’s atoning work as sufficient,
particularly by turning to human works or the Levitical system, is critical.
Although these warnings were addressed to Jewish Christians tempted to revert to
Judaism, the concept of working towards salvation, thus rejecting Christ’s
sufficiency, is universal for all Christians.

Hebrews 10:1--4, referencing Leviticus, explains that old covenant sacrifices
were shadows, while Christ’s sacrifice is definitive (Heb 10:10--14). John
Calvin warns, ``To reject Christ’s sacrifice is to trample underfoot the blood
of the covenant, for there remains no other sacrifice for sins'' (Calvin 1853,
245). Rejecting this truth risks peril (Heb 6:4--6, 10:26--31). Gareth Lee
Cockerill adds, ``To spurn Christ’s once-for-all sacrifice, as Hebrews warns,
is to reject the only means of atonement, incurring a judgment far greater than
under the old covenant'' (Cockerill 2012, 482). A stark warning is proffered:
unbelief or syncretism bars entry into God’s rest (Heb 3:7--19, 4:1--11).
Covenant fulfilment underscores the danger, as rejecting Christ’s sacrifice
negates the fulfiled covenant (Heb 10:29). Access to God’s presence is
jeopardised by reverting to works (Heb 10:19--22), and perfection is forfeited
by rejecting Christ’s sacrifice (Heb 10:26--31).

\section{The Motifs as Unifying Threads}
The motifs of rest, covenant fulfilment, access to God’s presence, and
perfection unify the six themes, each illuminating a facet of Christ’s work.
F.F. Bruce observes, ``The rest promised in Hebrews is the ultimate fulfilment
of God’s Sabbath, entered through faith in Christ, who fulfils the Old
Testament’s types and opens the heavenly sanctuary to believers'' (Bruce 1964,
96). John Owen adds, ``Christ’s priesthood and sacrifice grant access to God’s
presence, fulfilling the old covenant’s shadows and perfecting believers''
(Owen 1854, 3:89):
\begin{itemize}
    \item \textbf{Rest} (\textit{katapausis} and \textit{sabbatismos}, Heb 4:9)
    symbolises salvation’s goal, entered through Christ’s supremacy, the new
    covenant, and priesthood, dependent on faith, exhibited in holiness, and
    jeopardised by unbelief, framed by Old Testament references (e.g., Ps 95).
    \item \textbf{Covenant fulfilment} highlights Christ’s completion of Old
    Testament promises, surpassing the old covenant through His supremacy,
    priesthood, and new covenant, requiring faith and inspiring holiness, with
    warnings against rejecting the fulfilled covenant (e.g., Jer 31).
    \item \textbf{Access to God’s presence} emphasises salvation’s relational
    aspect, enabled by Christ’s mediation, secured through the new covenant and
    priesthood, accessed by faith, expressed in holiness, and lost through
    syncretism, rooted in Old Testament sanctuary imagery.
    \item \textbf{Perfection} underscores salvation’s completion, achieved by
    Christ’s supremacy, priesthood, and new covenant, received through faith,
    reflected in holiness, and forfeited by rejecting His atonement, contrasting
    with the Old Testament’s ineffective sacrifices.
\end{itemize}
Rest is experiential, symbolising salvation’s goal; covenant fulfilment is
theological, focusing on Old Testament fulfilment; access to God’s presence is
relational, emphasising intimacy; and perfection is soteriological,
highlighting completion. These motifs, grounded in the Old Testament, create a
cohesive narrative exalting Christ.

\section{Conclusion}
The Book of Hebrews weaves a rich theological tapestry grounded in Old Testament
imagery and fulfilled in Christ. Its core themes---Christ’s supremacy, the new
covenant, His eternal priesthood, faith and perseverance, the call to holiness
and worship, and the warning against rejecting His atonement---together present
a sustained argument for the sufficiency of Christ’s work and the necessity of
enduring faith. Interwoven throughout are the motifs of rest, covenant
fulfilment, access to God’s presence, and perfection, each deepening the
epistle’s message and unifying its structure.

Crucially, Hebrews presents not merely a rejection of the old covenant but a
transformative fulfilment of it. While certain elements---such as the Levitical
priesthood and repeated sacrifices---are rendered obsolete in light of Christ’s
once-for-all atonement (Heb 10:1–14), the theological arc of Scripture is not
broken but brought to maturity. The new covenant does not abolish God’s former
dealings but reveals their intended goal in Christ (Heb 8:6–13, 10:1). This dual
movement—both discontinuity and continuity—invites readers to interpret
Scripture not as static law but as dynamic promise, fulfilled through the
incarnate Son and now mediated to believers through faith.

Theologically, this calls readers to embrace Christ as the exclusive and
sufficient means of access to God, cautioning against any return to
performance-based righteousness or religious syncretism. Practically, Hebrews
exhorts believers to perseverance in trials, confident access to God’s presence
(Heb 4:16), reverent holiness (Heb 12:14, 28), and deep trust in the promises
yet to be fully seen (Heb 11:1). The motifs function not only as literary
devices but as discipleship tools, guiding believers into experiential rest,
covenantal identity, relational intimacy, and the hope of perfected salvation.

Ultimately, Hebrews compels a response: to trust fully in the sufficiency of
Christ’s finished work, to live in gratitude for God’s fulfilled promises, and
to await, with patient perseverance, the final consummation of the salvation
already secured.

\begin{thebibliography}{}
\bibitem{Athanasius2011}
Athanasius. \emph{On the Incarnation}. Translated by John Behr. Popular
Patristics Series 44. Yonkers, NY: St Vladimir’s Seminary Press, 2011.

\bibitem{Augustine1887}
Augustine. \emph{On the Spirit and the Letter}. Translated by Peter Holmes and
Robert Ernest Wallis. In \emph{Nicene and Post-Nicene Fathers}, First Series,
vol. 5, edited by Philip Schaff, 83--114. Buffalo, NY: Christian Literature
Publishing, 1887.

\bibitem{Bruce1964}
Bruce, F. F. \emph{The Epistle to the Hebrews}. New International Commentary on
the New Testament. Grand Rapids: Eerdmans, 1964.

\bibitem{Calvin1853}
Calvin, John. \emph{Commentaries on the Epistle to the Hebrews}. Translated by
John Owen. Edinburgh: Calvin Translation Society, 1853.

\bibitem{Cockerill2012}
Cockerill, Gareth Lee. \emph{The Epistle to the Hebrews}. New International
Commentary on the New Testament. Grand Rapids: Eerdmans, 2012.

\bibitem{Edwards1959}
Edwards, Jonathan. \emph{A Treatise Concerning Religious Affections}. In
\emph{The Works of Jonathan Edwards}, vol. 2, edited by John E. Smith. New
Haven: Yale University Press, 1959.

\bibitem{Hughes1977}
Hughes, Philip Edgcumbe. \emph{A Commentary on the Epistle to the Hebrews}.
Grand Rapids: Eerdmans, 1977.

\bibitem{Luther1968}
Luther, Martin. \emph{Lectures on Hebrews}. In \emph{Luther’s Works}, vol. 29,
edited by Jaroslav Pelikan, 109--241. St. Louis: Concordia, 1968.

\bibitem{OBrian2010}
O’Brien, Peter T. \emph{The Letter to the Hebrews}. Pillar New Testament
Commentary. Grand Rapids: Eerdmans, 2010.

\bibitem{Owen1854}
Owen, John. \emph{An Exposition of the Epistle to the Hebrews}. 7 vols.
Edinburgh: Johnstone \& Hunter, 1854--1855.

\bibitem{Schreiner2017}
Schreiner, Thomas R. \emph{Covenant and God’s Purpose for the World}. Short
Studies in Biblical Theology. Wheaton, IL: Crossway, 2017.

\bibitem{Sproul1985}
Sproul, R. C. \emph{The Holiness of God}. Wheaton, IL: Tyndale House, 1985.

\bibitem{Spurgeon1871}
Spurgeon, Charles H. ``The Brightness of His Glory.'' \emph{The Metropolitan
Tabernacle Pulpit Sermons} 17:1009--1016. London: Passmore \& Alabaster, 1871.

\bibitem{Wesley1755}
Wesley, John. \emph{Explanatory Notes upon the New Testament}. London: Epworth,
1755.
\end{thebibliography}

\end{document}
