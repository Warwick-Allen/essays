\documentclass[12pt]{article}
\usepackage{fontspec}
\defaultfontfeatures{Ligatures=TeX}
\setmainfont{Times New Roman}
\usepackage{geometry}
\usepackage{setspace}
\usepackage{enumitem}
\usepackage[hidelinks]{hyperref}
\usepackage{titlesec}
\usepackage{nowidow}
\usepackage[style=sbl,backend=biber]{biblatex}
\addbibresource{references.bib}

% Widow and orphan control: minimum 3 lines per page
\setnowidow[4]
\setnoclub[4]
\widowpenalty=10000
\clubpenalty=10000
\brokenpenalty=10000
\interlinepenalty=1000 % Discourage breaks within paragraphs

% Set page geometry for 25-mm (1-inch) margins on A4 paper
\geometry{a4paper, margin=25mm}

% Double spacing
\doublespacing

% Section formatting
\titleformat{\section}{\large\bfseries}{\thesection}{1em}{}
\titleformat{\subsection}{\normalsize\bfseries}{\thesubsection}{1em}{}

% SBL prefers ^^^^2026 over \ldots for ellipes
\def\ellipis{^^^^2026}

% Switch between Greek-based and Hebrew-based names and titles
\newif\ifusehebrew
%usehebrewtrue
\usehebrewfalse
\ifusehebrew
  \def\jesus{Yeshua}
  \def\christ{the Messiah}
  \def\Christ{The Messiah}
\else
  \def\jesus{Jesus}
  \def\christ{Christ}
  \def\Christ{Christ}
\fi

% Title, author, and date
\title{The Dominant Themes and Intersecting Motifs \\ in the Book of Hebrews}
\author{Warwick Allen}
\date{28 June 2025}

\begin{document}

\maketitle

\begin{abstract}
The Book of Hebrews is a theologically rich New Testament epistle aimed at
encouraging early Jewish Christians to remain steadfast in their faith amidst
pressures to revert to Judaism.
%
Six themes dominate the book, namely: \jesus{} \christ{}’s supremacy, the new
covenant, \christ{}’s priesthood, faith and perseverance, the call to holiness
and worship, and the warning against rejecting \christ{}’s sufficient atonement.
%
These are unified by four interwoven motifs: rest, covenant fulfilment, access
to God’s presence, and perfection.

Hebrews extensively relies on the Old Testament.
%
This grounds its claims, using Old Testament scriptures to demonstrate
\christ{}’s fulfilment of God’s redemptive plan.

The theme of \christ{}’s supremacy presents Him as superior to angels, Moses,
and the Levitical priesthood, fulfilling Old Testament promises and leading
believers into God’s rest.
%
The new covenant, superior to the Mosaic Law, offers forgiveness and direct
access to God, completing Old Testament shadows and enabling rest through faith.
%
\Christ{}’s eternal priesthood in the order of Melchizedek surpasses the
Levitical system, securing access to God’s presence and perfecting believers.
%
Faith and perseverance are urged through Old Testament examples, with unbelief
warned as barring entry to God’s rest. The call to holiness and worship reflects
gratitude for the new covenant, requiring a life aligned with God’s presence.
%
The warning against rejecting \christ{}’s atonement emphasises the peril of
relying on human works or the old sacrificial system, which jeopardises
redemption and rest.

The motifs unify these themes: rest symbolises salvation’s goal, covenant
fulfilment highlights \christ{}’s completion of Old Testament promises, access
to God’s presence emphasises relational intimacy, and perfection underscores
salvation’s completion.
%
Each motif, rooted in Old Testament imagery, reinforces \christ{}’s supremacy,
the new covenant’s efficacy, and the necessity of a faith kept pure from
syncretism with any form of works-based salvation.
%
Together, they create a cohesive narrative, exhorting believers to trust fully
in \christ{}’s finished work to enter God’s rest, access His presence, and
receive perfected salvation, fulfilling the Old Testament’s promises.
\end{abstract}

\section{Introduction}
The Book of Hebrews, a profound epistle in the New Testament, is a theological
masterpiece crafted to encourage early Jewish Christians to remain steadfast in
their faith in \jesus{} \christ{} amidst pressures to revert to Judaism. After
discussing the book’s Old Testament foundation, this essay explores the six
most dominant themes in Hebrews---namely, the supremacy of \christ{}, the new
covenant, the priesthood of \christ{}, faith and perseverance, the call to
holiness and worship, and the critical warning against failing to trust fully
in \christ{}’s sufficient atonement. It also examines how the motifs of rest,
covenant fulfilment, access to God’s presence, and perfection interweave these
themes, serving as unifying threads that underscore the book’s message of
salvation through faith in \christ{} alone.

This essay argues that Hebrews employs its themes and motifs to exhort believers
to trust fully in \christ{}’s finished work, fulfilling and surpassing the Old
Testament.

\section{Reliance on the Old Testament}
The Book of Hebrews heavily relies on the Old Testament to construct its
theological arguments, grounding its claims about \christ{} in the scriptures
familiar to its Jewish Christian audience. As John Calvin observes,
``The Epistle to the Hebrews is a kind of commentary on the Old Testament,
for it explains how the figures and shadows of the Law are fulfilled in
\christ{}, and how the promises given to the fathers are accomplished in Him''.
\footcite[7]{Calvin1853}
%
F.F. Bruce notes, ``The writer of Hebrews is saturated with the
Old Testament; he quotes it directly, alludes to it constantly, and assumes his
readers’ familiarity with its texts and themes to demonstrate the superiority
of Christ and His new covenant''. \footcite[14]{Bruce1964}
%
Hebrews includes direct quotations from the Septuagint, drawn from texts like
the Psalms, Jeremiah, and Leviticus (e.g., Ps 95 in Heb 3:7--11; Jer 31:31--34
in Heb 8:8--12).
%
These quotations, often introduced with phrases like ``as it is written'' or
``the Holy Spirit says,'' emphasise their authoritative role.
Beyond direct quotes, Hebrews frequently alludes to Old Testament figures (e.g.,
Melchizedek in Heb 7; Moses in Heb 3) and events (e.g., the wilderness wandering
in Heb 3--4), using them as types or shadows fulfilled in \christ{}.
Indirect references permeate the book’s discussions of the Law, sacrifices, and
priesthood, which assume familiarity with Old Testament institutions (e.g., Lev
16 in Heb 9).
%
This grounding in familiar scriptures bolstered the author’s appeal to Jewish
Christians facing temptation to either abandon \christ{}, or to see \christ{} as
an addition to the Levitical ordinances.

In terms of sheer volume, Matthew Capps notes that ``Hebrews contains 35 direct
quotations from the Old Testament''. \footcite[10]{Capps2015}
Gareth Lee Cockerill identifies twenty-eight distinct Old Testament passages
cited in the epistle, yielding ``a final number of 32 [Old Testament]
citations''. \footcite[14]{Cockerill2012}
Dana M. Harris reports that George Guthrie speaks of ``thirty-seven quotations,
[and] forty allusions'' to the Old Testament, underlining the letter’s
pervasive dependence on Scripture. \footcite[95]{Harris2021}
Harris summarises, ``The prevalence of the Old Testament is one of the most
striking features of the Epistle to the Hebrews''. \footcite[93]{Harris2021}

This pervasive Old Testament engagement anchors Hebrews’ argument that \christ{}
fulfills and surpasses the old covenant, making the Old Testament indispensable
to its message.

\section{The Supremacy of \Christ{}}
At the heart of Hebrews lies the theme of \christ{}’s supremacy, portraying Him
as superior to all figures and institutions of the Old Testament.
%
Hebrews 1:1--4 establishes \jesus{} as the eternal Son of God, the ultimate
revelation who surpasses angels, Moses, and the Levitical priesthood, fulfilling
Old Testament prophecies (e.g., Ps 2:7; 2 Sam 7:14).
%
Jonathan Edwards, preaching on \christ{}’s infinite greatness, declared that our
Saviour ``is infinitely great and high above all\ellipis He is higher than the
highest angels of heaven\ellipis angels themselves are as nothing before him.''
\footcite[103]{Edwards1981}
%
As the Creator and Sustainer of all things (Heb 1:3), \christ{} therefore alone
reigns at the Father’s right hand as the supreme Mediator.
%
In short, ``[\christ{}] has received a more excellent name than Moses'' and a
priesthood ``after the order of Melchizedek'' that far surpasses Aaron’s
(Heb 3:3; 7:17).
%
\Christ{}’s glory and authority are thus presented throughout Hebrews 1--4 as
utterly beyond any angel or human minister, securing full access to God’s
presence and the completeness of salvation.
%
\Christ{}’s supremacy, as the eternal Son and High Priest, not only fulfils the
Old Testament promises of a Messiah but also secures for believers the ultimate
rest that the old covenant could only foreshadow.

As the creator and sustainer of all things, seated at God’s right hand (Heb
1:3), \christ{} is the sole mediator of salvation.
%
The motif of rest ties directly to this theme, as \christ{} leads believers into
God’s true rest, surpassing the temporary rest of the Promised Land under Joshua
(Heb 4:8--10, referencing Ps 95).
%
\Christ{}’s divine authority ensures a rest surpassing Joshua’s, as His eternal
mediation secures what temporary leaders could not.
Covenant fulfilment underscores \christ{}’s role as the culmination of Old
Testament promises (Heb 1:1--2), as Augustine notes, ``Christ is the end of the
law, the prophets, and the patriarchs; in Him all things are fulfilled''.
\footcite[14.24]{Augustine1887}
Access to God’s presence is enabled by His exalted position (Heb 4:16), and
perfection highlights \christ{} as the pioneer of salvation, made ``perfect
through suffering'' (Heb 2:10, alluding to Old Testament suffering servant
imagery of, e.g., Isa 50:4--9 \& 52:13--53:12).

The Book of Hebrews strongly affirms the divinity of \christ{}, a truth that
undergirds his supremacy over all things. From the outset, the author presents
\jesus{} as the Son of God, through whom God created the universe and who is
``the exact representation of his being'' (Heb 1:2-3).
%
This language identifies the Son as the sustainer of all things and the one who
has purified sins, roles attributed to God himself. Moreover, in Hebrews 1:8,
the author quotes Psalm 45:6-7, where God addresses the Son as ``God,''
affirming his divine status and eternal reign.
%
The epistle also emphasises \christ{}’s unchanging nature in Hebrews 13:8,
stating that ``\jesus{} \christ{} is the same yesterday and today and forever,''
a declaration that aligns with God’s immutability (e.g., Mal 3:6). As scholar
William Lane observes, ``The affirmation of the Son’s deity is integral to the
argument of Hebrews, establishing his unique qualifications as the mediator of
salvation''. \footcite[15]{Lane1991}
%
Thus, through these scriptural references and theological assertions, Hebrews
unequivocally establishes \christ{}’s divinity as central to his supreme
authority and salvific work.

Flowing naturally from \christ{}’s supremacy is the establishment of a new
covenant, for His exalted status not only fulfils the promises of the old but
inaugurates a superior covenantal relationship that renders the former obsolete.

\section{The New Covenant}
Hebrews contrasts the old covenant, rooted in the Mosaic Law, with the new
covenant established through \christ{}’s sacrifice. This new covenant is
superior, offering forgiveness and direct access to God (Heb 8:6--13, quoting
Jer 31:31--34). John Owen affirms, ``The new covenant, as foretold by Jeremiah
and established by \christ{}, is the perfection of God’s promises, whereby sins
are forgiven, and the law is written on the heart, making the old covenant
obsolete''. \footcite[4:123]{Owen1854}
%
Hebrews 9:15 describes \christ{} as its mediator,
replacing the repetitive sacrifices of the old system (Lev 16) with His
once-for-all atonement. The motif of rest connects here, as the new covenant
enables believers to enter God’s rest through faith (Heb 4:3--5). Covenant
fulfilment is central, as the new covenant completes the Old Testament’s
temporary provisions (Heb 8:13). Access to God’s presence is a hallmark,
symbolised by the torn temple veil (Heb 10:19--22, referencing Old Testament
sanctuary imagery).
%
Perfection is achieved through forgiveness, perfecting believers’ consciences
where the old covenant failed (Heb 10:1--4, 14), as R.C.  Sproul notes, ``The
new covenant in Hebrews is the fulfillment of Jeremiah’s prophecy, where
Christ’s once-for-all sacrifice replaces the repeated offerings of the old
covenant''. \footcite[45]{Sproul1985}

\subsection{Debates over Covenant Continuity and Discontinuity}
The portrayal of the new covenant in Hebrews has sparked significant scholarly
debate regarding its relationship to the old covenant. This debate, which
centres on the extent of continuity and discontinuity between the two covenants,
is crucial for understanding how Hebrews presents the new covenant as superior
and how this superiority relates to the epistle’s broader themes and motifs.
While the previous section established the new covenant’s role in offering
forgiveness and direct access to God, the question remains: does this new
covenant represent a radical break from the old, or does it fulfil and perfect
what the old covenant foreshadowed?

Hebrews itself provides seemingly contrasting signals. On one hand, Hebrews 8:13
declares that ``by calling this covenant \lq{}new,\rq\ he has made the first one
obsolete; and what is obsolete and outdated will soon disappear,'' suggesting a
sharp discontinuity.
%
This perspective is echoed by scholars like Thomas Schreiner, who argues that
Hebrews presents a theology of covenantal displacement, in which the Mosaic
covenant becomes nullified in light of \christ{}’s superior priesthood and
sacrifice. \footcite[204]{Schreiner2017}
%
Such a view underscores the radical nature of \christ{}’s work, rendering the
old covenant’s sacrificial system entirely obsolete and emphasising the urgency
of embracing the new covenant fully.
%
This interpretation aligns with Hebrews’ stark warnings against reverting to the
old system (e.g., Hebrews 10:26--31), as doing so would mean rejecting the only
sufficient means of atonement.

On the other hand, Hebrews also portrays the new covenant as the fulfilment of
promises embedded in the old covenant, suggesting a deep continuity. Peter
O’Brien notes that while the old covenant is depicted as insufficient for
salvation, it is never discarded as irrelevant. Instead, the new covenant is
presented as the divinely intended consummation of the old, particularly through
the prophecy of Jeremiah 31:31--34, which Hebrews quotes at length in 8:8--12.
\footcite[296]{OBrien2010}
This motif of fulfilment is further evident in Hebrews
10:1, where the law is described as ``only a shadow of the good things that are
coming---not the realities themselves.'' Thus, the new covenant does not abolish
the old arbitrarily but brings its sacrificial, priestly, and prophetic elements
to their intended completion in \christ{}. John Calvin captures this dual
perspective, stating, ``The old covenant was not contrary to the gospel, but was
its cradle; in \christ{}, the shadows are removed and the truth shines forth''.
\footcite[45]{Calvin1853}

This scholarly debate is not merely academic; it profoundly shapes how we
understand the epistle’s dominant themes and intersecting motifs.
%
If the new covenant is seen as a sharp discontinuity, it highlights the
transformative nature of \christ{}’s work and the obsolescence of the old
system, reinforcing the theme of \christ{}’s supremacy and the peril of
rejecting His sufficient atonement, and intensifies warnings like Hebrews 6:4-6,
where rejecting \christ{} leaves no alternative atonement.
%
This perspective also sharpens the motif of rest, as entry into God’s rest
becomes exclusively tied to faith in \christ{} rather than adherence to the old
covenant’s practices.
%
Conversely, viewing the new covenant as a deep continuity enriches our
understanding of motifs like covenant fulfilment and perfection.
%
It shows how the old covenant’s temporary provisions—such as the Levitical
priesthood and repeated sacrifices—find their substance and completion in
\christ{}’s eternal priesthood and once-for-all sacrifice.
%
This fulfilment motif underscores that \christ{} does not discard the old
covenant but perfects it, achieving what it could only foreshadow.

Moreover, the way one interprets the relationship between the covenants
influences how believers understand their identity and practice under the new
covenant. A discontinuity view might lead to a greater emphasis on the
distinctiveness of Christian worship and ethics, free from the old covenant’s
rituals. In contrast, a continuity view could encourage believers to see their
faith as the true realisation of Old Testament worship, with \christ{} as the
ultimate High Priest and sacrifice. This perspective deepens the call to
holiness and worship (Heb 12:14, 28), as believers are invited to live in
grateful response to the fulfilled promises of God.

Ultimately, while Hebrews clearly presents the new covenant as superior and the
old as obsolete, the epistle also weaves a narrative of fulfilment that honours
the old covenant’s role in God’s redemptive plan.
%
This nuanced interplay between continuity and discontinuity enriches our
understanding of the epistle’s themes, particularly the supremacy of \christ{}
and the perfection He achieves for believers.
%
It also reinforces the urgency of the warning against rejecting \christ{}’s
atonement, as to spurn the fulfilled covenant is to forfeit the very rest,
access, and perfection that the old covenant could never provide.

\section{The Priesthood of \Christ{}}
\Christ{}’s role as the eternal High Priest in the order of Melchizedek,
surpassing the Levitical priesthood, is a dominant theme.
%
Hebrews 7:23--28, referencing Ps 110:4, highlights His eternal priesthood, with
His single sacrifice perfecting believers forever (Heb 10:11--14).
%
Philip Edgcumbe Hughes states, ``The
priesthood of Melchizedek, as expounded in Hebrews, reveals \christ{} as the
eternal priest who, by His own blood, secures an eternal redemption''.
\footcite[258]{Hughes1977}
This priesthood grants access to God’s presence (Heb 4:14--16),
central to the motif of rest, as the \emph{sabbatismos} of Heb 4:9--10
reflects God’s Sabbath rest, shared through \christ{}’s priestly work (Gen 2:2).

Cyril of Jerusalem (c. 315–386) discusses how \christ{}’s divine-human nature is
tied to his eternal priesthood. In his Catechetical Lectures he explains that
\jesus{} was ``eternally anointed by the Father to His High-Priesthood on behalf
of men'', meaning that \christ{}’s role as High Priest comes directly from God
and is set for all eternity.
%
Cyril goes on to emphasise that \christ{} ``is a High Priest, whose priesthood
passes not to another'', underscoring that \jesus{} alone offers the
once-for-all sacrifice for sin. \footcite[Lecture 12]{Cyril1894}
%
This echoes the book of Hebrews: as our perfect God-man High Priest, \christ{}
has entered the heavenly sanctuary to intercede for us and grant us direct
access to God, accomplishing what the old priestly sacrifices could only
foreshadow (cf.  Heb 4:14–16).

Covenant fulfilment is evident, as \christ{}’s priesthood completes the Aaronic
system (Heb 7:11--22). Access to God’s presence is secured by His mediation (Heb
10:19), and perfection is achieved, as His priesthood makes believers complete
(Heb 7:19, 28).

\section{Faith and Perseverance}
Hebrews emphasises faith and perseverance, particularly in trials.
%
Chapter 11 recounts Old Testament figures who, despite significant failures and
moral shortcomings, exemplified enduring faith, encouraging believers to
persevere (Heb 10:19--39).
%
Martin Luther observes, ``Faith, as Hebrews 11 teaches, is the substance of
things hoped for, and by it the saints of old trusted God’s promises, looking
forward to Christ''. \footcite[26]{Luther1968}
%
The motif of rest is linked, as Heb 3:12--19, citing Ps 95, warns that unbelief
prevents entry into God’s rest, while Heb 4:2--3 assures that faith secures it.
Abraham’s faith (Heb 11:8--10) anticipates the rest fulfilled in \christ{} (Heb
4:9), illustrating perseverance toward God’s promise.
%
Abraham’s faith amid uncertainty (Heb 11:8) models perseverance for believers
facing persecution.
%
John Wesley adds, ``The faith of the patriarchs, as recounted in Hebrews, is our
example to persevere in trusting Christ, for without faith it is impossible to
please God''. \footcite[Heb 11:6]{Wesley1755}
%
Covenant fulfilment connects, as faith in \christ{}’s work secures Old Testament
promises (Heb 11:39--40).
%
Access to God’s presence is accessed through faith (Heb 4:16), and perseverance
ensures continued access (Heb 10:22).
%
Perfection is received by faith, as believers await completion through \christ{}
(Heb 11:40).

\section{The Call to Holiness and Worship}
Hebrews urges believers to embrace holiness and worship as vital responses to
\christ{}’s redemptive work.
%
Hebrews 12:14 exhorts, ``be holy; without holiness no one will see the Lord,''
while Hebrews 12:28-29 calls for worship ``with reverence and awe.''
%
These commands are not mere ideals but practical expressions of faith.
%
Jonathan Edwards captures this, noting, ``True holiness, as urged in Hebrews, is
the fruit of faith in Christ’s redemption, whereby we live to God’s glory and
worship Him in reverence''. \footcite[2.3]{Edwards1959}

\emph{Holiness} entails ethical living, compassion, and peacemaking.
%
Practically, believers can reflect this by making honest choices at work,
showing kindness to the marginalised, or mediating disputes with grace (Heb
12:14).
%
This flows from \christ{}’s sanctifying sacrifice (Heb 10:14).
%
Holiness reflects the perfection \christ{} imparts (Heb 10:14), aligning
believers with their completed status.
%
R.C. Sproul underscores its urgency: ``Holiness is not optional for the
Christian; Hebrews 12:14 declares it essential to see the Lord''.
\footcite[112]{Sproul1985}
%
Without it, we cannot fully enter God’s presence.

\emph{Worship}, meanwhile, transcends rituals, permeating daily life through
gratitude, service, and awe.
%
Simple acts---thanking God for a meal, serving a neighbour, or marvelling at a
sunset---become offerings of praise (Heb 13:15-16).
%
Engaging Scripture or creation deepens reverence (Heb 12:28).

These practices interweave with Hebrews’ motifs: holiness and worship bring rest
(Heb 4:9), fulfil the new covenant (Heb 8:10-12), grant access to God
(Heb 12:14), and reflect \christ{}’s perfection (Heb 10:14).
%
Yet, they stem from grace, not obligation (Heb 10:10).
%
Edwards and Sproul affirm that holiness and worship, rooted in \christ{}’s
sufficiency, transform believers’ lives, fostering peace, integrity, and
intimacy with God.
%
But failing to recognise that holiness and acts of worship stem from grace and
are rooted in \christ{}’s sufficiency, and thinking that they somehow help
towards our redemption, is, in fact, rejecting the sufficiency of \christ{}'s
atonement.

\section{Warning Against Rejecting \Christ{}’s Sufficient Atonement}
The warning against failing to accept \christ{}’s atoning work as sufficient,
particularly by turning to human works or the Levitical system, is critical.
%
Although these warnings were addressed to Jewish Christians tempted to revert to
Judaism, the concept of working towards salvation, thus rejecting \christ{}’s
sufficiency, is universal for all Christians.

Hebrews 10:1--4, referencing Leviticus, explains that old covenant sacrifices
were shadows, while \christ{}’s sacrifice is definitive (Heb 10:10--14). John
Calvin warns, ``To reject Christ’s sacrifice is to trample underfoot the blood
of the covenant, for there remains no other sacrifice for sins''.
\footcite[245]{Calvin1853}
Rejecting this truth risks peril (Heb 6:4--6, 10:26--31).
%
Gareth Lee Cockerill adds, ``To spurn Christ’s once-for-all sacrifice, as
Hebrews warns, is to reject the only means of atonement, incurring a judgment
far greater than under the old covenant''. \footcite[482]{Cockerill2012}
%
A stark warning is proffered:
unbelief or syncretism bars entry into God’s rest (Heb 3:7--19, 4:1--11).
%
Covenant fulfilment underscores the danger, as rejecting \christ{}’s sacrifice
negates the fulfilled covenant (Heb 10:29).
%
Access to God’s presence is jeopardised by reverting to works (Heb 10:19--22),
and perfection is forfeited by rejecting \christ{}’s sacrifice (Heb 10:26--31).
%
Today, this warning applies to reliance on moralism, echoing the legalism
Hebrews rejects.

\section{The Motifs as Unifying Threads}
The motifs of rest, covenant fulfilment, access to God’s presence, and
perfection unify the six themes, each illuminating a facet of \christ{}’s work.
F.F. Bruce observes, ``The rest promised in Hebrews is the ultimate fulfilment
of God’s Sabbath, entered through faith in Christ, who fulfils the Old
Testament’s types and opens the heavenly sanctuary to believers''.
\footcite[96]{Bruce1964}
%
John Owen adds, ``Christ’s priesthood and sacrifice grant access to God’s
presence, fulfilling the old covenant’s shadows and perfecting believers''
\footcite[3:89]{Owen1854}:
\begin{itemize}
    \item \textbf{Rest} (\emph{katapausis} and \emph{sabbatismos}, Heb 4:9)
    symbolises salvation’s goal, entered through \christ{}’s supremacy, the new
    covenant, and priesthood, dependent on faith, exhibited in holiness, and
    jeopardised by unbelief, framed by Old Testament references (e.g., Ps 95).
    \item \textbf{Covenant fulfilment} highlights \christ{}’s completion of Old
    Testament promises, surpassing the old covenant through His supremacy,
    priesthood, and new covenant, requiring faith and inspiring holiness, with
    warnings against rejecting the fulfilled covenant (e.g., Jer 31).
    \item \textbf{Access to God’s presence} emphasises salvation’s relational
    aspect, enabled by \christ{}’s mediation, secured through the new covenant
    and priesthood, accessed by faith, expressed in holiness, and lost through
    syncretism, rooted in Old Testament sanctuary imagery.
    \item \textbf{Perfection} underscores salvation’s completion, achieved by
    \christ{}’s supremacy, priesthood, and new covenant, received through faith,
    reflected in holiness, and forfeited by rejecting His atonement, contrasting
    with the Old Testament’s ineffective sacrifices.
\end{itemize}
Rest is experiential, symbolising salvation’s goal; covenant fulfilment is
theological, focusing on Old Testament fulfilment; access to God’s presence is
relational, emphasising intimacy; and perfection is soteriological,
highlighting completion. These motifs, grounded in the Old Testament, create a
cohesive narrative exalting \christ{}.

\section{Conclusion}
The Book of Hebrews weaves a rich theological tapestry grounded in Old Testament
imagery and fulfilled in \christ{}.
%
Its core themes---\christ{}’s supremacy, the new covenant, His eternal
priesthood, faith and perseverance, the call to holiness and worship, and the
warning against rejecting His atonement---together present a sustained argument
for the sufficiency of \christ{}’s work and the necessity of enduring faith.
%
Interwoven throughout are the motifs of rest, covenant fulfilment, access to
God’s presence, and perfection, each deepening the epistle’s message and
unifying its structure.

Crucially, Hebrews presents not merely a rejection of the old covenant but a
transformative fulfilment of it. While certain elements---such as the Levitical
priesthood and repeated sacrifices---are rendered obsolete in light of
\christ{}’s once-for-all atonement (Heb 10:1–14), the theological arc of
Scripture is not broken but brought to maturity. The new covenant does not
abolish God’s former dealings but reveals their intended goal in \christ{} (Heb
8:6–13, 10:1).
%
This dual movement—both discontinuity and continuity—invites readers to
interpret Scripture not as static law but as dynamic promise, fulfilled through
the incarnate Son and now mediated to believers through faith.

Theologically, this calls readers to embrace \christ{} as the exclusive and
sufficient means of access to God, cautioning against any return to
performance-based righteousness or religious syncretism. Practically, Hebrews
exhorts believers to perseverance in trials, confident access to God’s presence
(Heb 4:16), reverent holiness (Heb 12:14, 28), and deep trust in the promises
yet to be fully seen (Heb 11:1). The motifs function not only as literary
devices but as discipleship tools, guiding believers into experiential rest,
covenantal identity, relational intimacy, and the hope of perfected salvation.

Ultimately, Hebrews compels a response: to trust fully in the sufficiency of
\christ{}’s finished work, to live in gratitude for God’s fulfilled promises,
and to await, with patient perseverance, the final consummation of the salvation
already secured.

\printbibliography[title=Bibliography]

\end{document}
